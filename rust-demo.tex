\documentclass{acm_proc_article-sp}

\begin{document}
\title{Demo: The Rust Language and Type System}
\numberofauthors{2}
\author{
% 1st. author
\alignauthor
Felix S. Klock II\\
    \affaddr{Mozilla Research}\\
    \email{pnkfelix@mozilla.com}
% 2nd. author
\alignauthor
Nicholas Matsakis\\
    \affaddr{Mozilla Research}\\
    \email{nmatsakis@mozilla.com}
}
\maketitle

\section{Abstract}

\emph{Rust} is a new programming language for developing reliable and
efficient systems.  It is designed to support concurrency and
parallelism in building applications and libraries that take full
advantage of modern hardware.
 Rust's static type system is safe\footnote{Type soundness has not
  yet been formally proven for Rust, but type soundness is an
  explicit design goal for the language.} and expressive
and provides strong guarantees about isolation,
concurrency, and memory safety.

Rust also offers a clear performance model, making it easier to
predict and reason about program efficiency.  One important way it
accomplishes this is by allowing fine-grained control over memory
representations, with direct support for stack allocation and
contiguous record storage.  The language balances such controls with
the absolute requirement for safety: Rust's type system and runtime
guarantee the absence of data races, buffer overflows, stack overflows,
and accesses to uninitialized or deallocated memory.

In this demonstration, we will briefly review the language features
that Rust leverages to accomplish the above goals, focusing in
particular on Rust's advanced type system, and then show a collection
of concrete examples of program subroutines that are efficient, easy
for programmers to reason about, and maintain the above safety
property.

If time permits, we will also show the current state of \emph{Servo},
Mozilla's research web browser that is implemented in Rust.

\keywords{Rust, Systems programming, Memory management, Affine type systems}

\section{Demonstration Outline}

{}

Planned topics for the demonstration include:
\begin{itemize}
\item Memory safety in Rust
\item Ownership and linear types
\begin{itemize}
\item Data Allocation
\item Deallocation and (user-defined) Destructors
\item Inherited mutability
\item Moving and sending data
\end{itemize}
\item Borrowing and freezing of mutable data structures
\begin{itemize}
\item The necessity of (controlled, temporary) aliasing
\item Explicit and implicit lifetime parameters
\end{itemize}
\item Discussion of design space
\begin{itemize}
\item Comparison against other work
\item Moving values based on type, vs {\tt move} keyword
\item Linear mutable values
\item Lexically scoped lifetimes
\end{itemize}
\end{itemize}
\end{document}
